\input respnotes

\def\H{{\bf H}}

\title A review of variational Gaussian wave packets

This will summarise the work that has been done from the time of Heller to the present day.  It will take the viewpoint of frames and stability.

Notation.  A superposition of~$r$ multimode wave packets, each with~$n$ modes, is~$r×n$.  (The~$×n$ can be omitted for a single mode.)  MCTDH matrices use~$⊗$, so a full tensor product of wave packets over 3 modes might be~$2⊗2⊗2$, or~$2³$.  (Evaluating these as integer expressions gives you a count of the parameters.)  A Hilbert space for a single degree of freedom, expanded over~$r$ basis states that are not gaussian wave packets, is written~$r×\H$.  So the normal MCTDH has structure~$(n₁×\H)⊗…⊗(n_f×\H)$.  An arbitrary expansion is~$\H$, so~$\Hⁿ$ is a general state with~$n$ degrees of freedom.

\section Dirac and Frenkel

Dirac-Frenkel dynamics necessarily conserves the norm of the ket and~$〈H〉$.  Dirac-Frenkel variational coefficients conserve the norm, whatever the amplitudes are forced to do \cite{jcp-136-014109}.

Kay explains that there are several forms of the variational principle
\cite{cpx-137-165}.  In all cases, there is a ket~$|ψ(z)〉$ that
sits in a complex vector space, and depends on variational
parameters~$z$.  In principle, we always want to select~$dz$ so
that~$|ψ(z+dz)〉=|ψ(z)〉+|dψ〉$ in a Hilbert-space least squares
sense.  This occurs when the residual~$i\hbar|ψ'(z)〉·dz-H|ψ(z)〉·dt$
is Hilbert-space orthogonal to the column space of~$|ψ'(z)〉$.  The
simplest case is when~$z$ is complex, and~$|ψ〉$ is analytic in
each parameter, and orthogonality reduces
to
$$〈D_iψ(z)|\left(i\hbar|ψ'(z)〉·dz-H|ψ(z)〉·dt\right)=0\eqno{\tag\eqi}$$
for all~$i$.  This is how Frenkel did it \cite{1934-Frenkel-Wave}, and he's OK.

This doesn't work when~$z$ is real, and therefore~$dz$ is real.
Taking real and imaginary parts, there are twice as many Equations~\eqi\
as there are unknowns.  What we should do is take real and imaginary
parts of the residual and the~$|D_iψ(z)〉$ as expanded over some
orthonormal basis, to get a real least squares problem where more
equations than unknowns is business as usual.  What the twits who
invented quantum mechanics did instead was to arbitrarily demand that
we set the real part of Equations~\eqi\ to zero following
the “least-action form” \cite{cpx-137-165[3]}, or their imaginary
parts to zero following McLachlan \cite{cpx-137-165[2]}.  This has
led to lots of point scoring and name calling regarding which is
better.

\section Heller

\section MCTDH

MCTDH is \dots

The “stays orthogonal” constraint can be expressed as the single-particle modes in each subspace obey some Hamiltonian dynamics.

Follow up cpl-368-502[1:7 10 11 12 13:15 21], jcp-111-2927[12]

The original idea \cite{jcp-111-2927} was to treat the system with normal MCTDH, and the bath with a multimode gaussian wave packet, with structure~$(n₁×\H)⊗…⊗(n_{f-1}×\H)⊗(n_f×m)$.  Motivated by molecules in solution.  Assume the solvents have a small range of motion, so that their potential is nearly harmonic.  This paper derived the variational equations for a density matrix and master equation.

Worth and Burghardt \cite{cpl-368-502} combined variational wave packets with MCTDH.  The idea was simply to use gaussian wave packets as the orbitals~$φⁿ_i$ in that method.  (Note that this arranges the tensor products differently than in the multimode coherent state version.)  The hope was to use multidimensional wave packets, with each set of parameters covering several degrees of freedom.  This was applied to a Henon-Heiles potential, and reproduced a numerically exact autocorrelation function.  It's unclear why this was numerically stable.

This used structures~$(5×2)²$ to~$(25×2)²$, in a phase space of volume~?.
{\bf Check out who cited it.}

This managed to superpose 625 wavepackets in 4D space.  {\bf How did they deal with the lack of orthogonality?}  Wave packets were deleted when they escaped to large distance.

Noted that tunnelling happens by lots of energy being transferred to a subset of the wave packets.

Another application in \cite{jcp-119-5364}.

\section Frozen trajectories

\cite{jcp-132-244111}

Habershon \cite{jcp-136-014109} deleted redundant wave packets to maintain full rank, used Eherenfest type trajectories.

\section Regularisation

Kay cph-137-165 thought about SVDs and rank-deficiency, and noted that truncating the small sws and taking a pseudoinverse helped in some cases.  He didn't directly consider gaussian wave packets.  Follow up his refs and cites.  (This is actually a discrete problem, not a rank-deficient one 2010-Hansen-Discrete).  

Habershon \cite{jcp-136-014109} propagated an ensemble of wave packets using Eherenfest type trajectories, but forced the variational coefficients of the redundant ones to zero.  He substituted packets in and out as the condition of the expansion operator changed.  This paper treated a double well potential.

The initial amplitudes were sampled from the Wigner distribution for the initial state, no justification for why it wasn't Q.  For a quartic oscillator state, Wigner sampling would generate lots of wave packets that don't overlap the state.  This used a pseudoinverse that truncated small singular values.  Habershon's results diverge in a very similar way to our quartic oscillator results.  He attributes this to conservation of energy failing, but he doesn't think through the stiffness and high energy state ideas.  His stabilisation approach works.  He claims that it works because he is constantly re-expanding the state over a new set of active wave packets.  It's plausible that a parasitic solution in one expansion is well-behaved in the next expansion.

{\bf Discuss section III of jcp-136-014109 in detail.}

\section Index of potentials treated

Henon-Helies
double well

\section To look at

1968-Klauder-Fundamentals
2014-Ali-Coherent
cpl-118-558
cpl-165-73
cpl-319-674
cpl-368-502
cpx-137-165
cpx-304-103
http://pubs.acs.org/doi/pdf/10.1021/ar00072a002
http://pubs.rsc.org/is/content/articlelanding/2007/cp/b700297a/unauth#!divAbstract
https://doi.org/10.1080/00268970802258609
https://doi.org/10.1080/00268978600100551
https://doi.org/10.1090/S0025-5718-04-01685-0 
https://link.springer.com/article/10.1007%2Fs00791-006-0019-8?LI=true
ijc-47-75
jcp-110-5526
jcp-111-2927
jcp-112-6113
jcp-117-4738
jcp-119-1961
jcp-119-5364
jcp-121-9247
jcp-128-054102
jcp-129-174104
jcp-130-134113
jcp-132-244111
jcp-136-014109[1:17]
jcp-62-1544
jcp-64-63
jcp-65-4979
jcp-67-2017
jcp-71-3383
jcp-75-246
jcp-75-2923
jcp-80-3123
jcp-83-3009
jcp-83-3009[1:10]
jcp-84-227
jcp-84-3250
jcp-84-6293
jcp-85-4129
jcp-87-5302
jcp-87-5987
jcp-90-3060
jcp-91-170
jcp-93-3919
jcp-94-2723
jcp-96-4266
jpa-19-2559
njp-10-115005
pna-14-187
pra-13-2226
pra-73-033619
prb-73-125112
prx-131-2766
prx-177-1882
rmp-2-221
ssc-31-3027
xxx-physics-0510076

\section Rejected as irrelevant

jcp-117-4738 (only one wave packet, $r=1$)